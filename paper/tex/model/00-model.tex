\section{System Model}

\subsection{Task Model}
We consider a task set $\tau$ of ${n}$ sporadic parallel real time tasks,
$\tau = \{\tau_1,\tau_2, ..., \tau_n\}$. A task $\tau_i$ in the task
system generates a potentially infinite number of jobs, each arriving
no less than $T_i$ time units after the previous job and has a constrained
deadline $D_i$ where $D_i \leq T_i$.  Each task $\tau_i \in \tau$ is
considered as a parallel task and is represented as a directed acyclic
graph (DAG) ${G_i}$, the set of DAGs is given as ${G = \{G_1, G_2,
  ..., G_n\}}$. %An example DAG is shown in Fig.~\ref{fig:dag}.

%\begin{figure}[!h]
%  \centering
%  \includegraphics[width=\columnwidth]{DAG}
%  \caption{An Example of a Parallel Real-time Task}
%  \label{fig:dag}
%\end{figure}

Each node ${v \in V_i}$ of a DAG ${G_i = (V_i, E_i)}$ represents the
execution of a thread and an ${e \in E_i}$ edge represents a
dependency between two nodes. A dependency between two nodes indicates
that a node is ready to execute only upon the completion of all its
predecessor nodes. We consider a source as a node with no incoming
arcs and a sink as node with no outgoing arcs. For the sake of
simplicity, we assume a task has only one source and sink node. In
practice, this not hard to achieve as a source and sink can be added
to the task with $0$ execution and not change the task dependencies.

Nodes of a task have an associated worst-case execution time. For any
given task, we define a \textit{critical path} $\lambda_i$ of a 
task $\tau_i$ as the longest execution time path that starts from
source and ends at the sink. \textit{Critical path length} ($L_i$) is
defined as the sum of execution times of all nodes along the critical
path $\lambda_i$ of task $\tau_i$.  Workload $C_i$ of
a task $\tau_i$ is defined as the sum of worst-case execution time of
all nodes in the DAG task. 


\subsection{Federated Scheduler}
For a task set $\tau$ , the federated scheduling algorithm works as
follows. We first divide the task sets into two disjoint sets
$\tau_{high}$  and $\tau_{low}$. $\tau_{high}$ contains all tasks with
high utilization (i.e. $u_i > 1$) and $\tau_{low}$ contains all
remaining low utilization tasks. Each task in $\tau_{high}$ is
assigned $m_i$ dedicated cores (no other task is executed on these
cores), where: \begin{equation}\label{eq:m} m_i = \left\lceil \frac{C_i - L_i}{D_i - L_i}
\right\rceil \end{equation}

We use $m_{high} = \sum_{\tau_i \in \tau_{high}} m_i$ to denote the total
number of cores assigned to high-utilization tasks. We assign
the remaining cores to all low-utilization tasks $\tau_{low}$, denoted
as ${m_{low} = m - m_{high}}$. The federated scheduling algorithm admits
the task set ${\tau}$, if $m_{low}$ is non-negative and all tasks in
$\tau_{low}$ are schedulable sequentially.  

After a valid core allocation, runtime scheduling proceeds as
follows. Any greedy (work-conserving) parallel scheduler can be used
to schedule a high-utilization task $\tau_i \in \tau_{high}$ on its
assigned $m_i$ cores. Informally, a greedy scheduler is one that never
keeps a core idle if some node is ready to execute. 

All low-utilization tasks are treated and executed as though they are
sequential tasks and any multiprocessor scheduling algorithm (such as
partitioned EDF, or various rate-monotonic schedulers) can be used to
schedule all the low-utilization tasks on the allocated $m_{low}$
cores. We can safely treat low-utilization tasks as sequential tasks
since $C_i \le D_i$ and parallel execution is not required to meet their
deadlines. 

\subsection{Processing Model}

In this work, for each core, we assume a dedicated direct mapped
instruction cache. We assume a time-compositional architecture\addcite,
where memory and execution demand are separable. Copying a block of 
main memory to cache memory requires ${\mathbb{B}}$ cycles, commonly
referred to as the the block reload time (BRT). If multiple cores share
the same processing platform their cache contents do not interfere with
one another. The impact of a shared cache between cores is not considered.


\section{Proposed Changes to the Directed Acyclic Graph Model of Parallel Systems}

For a DAG ${G = (V, E)}$ representing a parallel task, each node ${v_i \in V}$ represents
the release, execution, and termination of a single thread within one task 
\addcite. In the existing model, the only relationship between thread releases and the executable object they execute is the worst-case execution time of the node. Two nodes ${v_i, v_j \in V}$ may represent two threads executing the same object (possibly on different processors).
\\
\\
\emph{ct-1.) A figure is needed here, illustrating the existing DAG model}
\\

To take advantage of instruction cache reuse, we propose a simple modification to the DAG
model. Where possible, distinct nodes that represent the execution of
the same object are collapsed into a single node. To accommodate collapse,
nodes are identified by their executable object, and a new attribute is added to every node
indicating the number of threads which will be executed over the object.
\\
\\
\emph{ct-2.) A figure is needed here, illustrating collapse from ct-1 to ct-2}
\\

% \begin{figure}[h]
%   \centering
%   \subfloat[Before Collapse]{%
%     \includegraphics{Pictures/before-collapse}
%     \label{fig:before-collapse}
%   }
%   \subfloat[After Collapse]{%
%     \includegraphics{Pictures/after-collapse}
%     \label{fig:after-collapse}
%   }
%   \caption{DAG Node Change from Thread to Code Segment}
%   \label{fig:dag-change}
% \end{figure}

Figure~\ref{fig:dag-change} illustrates the proposed change. Nodes in
Figure~\ref{fig:before-collapse} are labeled with the code segment
they are associated with. Two nodes of ${B}$ are collapsed in
Figure~\ref{fig:after-collapse}, where each node is attributed with
the number of threads executed over the code segment.

A goal of this work is to bring the inter-thread cache benefit \addcite to parallel DAG tasks. As a first-step, two requirements are placed on nodes of the graph.
\begin{description}
\item[R1] All executable objects must fit entirely within the cache.
\item[R2] No two instructions of an executable object may evict one another.
\end{description}

Requirements R1 and R2 may be met for any executable object by repeatedly dividing
the object source that result in objects larger than the cache into separate code segments, carefully recompiling those code segments to maximize cache use, and replacing the original node with a serial set of nodes.
\\
\\
\emph{ct-3 A figure is needed to illustrate the transformation from one over-sized node, to multiple correctly sized nodes}
\\

In the established model \addcite, each node ${v \in V_i}$ is characterized
with a single worst case execution time for a single thread. We
propose that each node's WCET is characterized by a function
${c_i(n)}$ where ${n}$ is the number of threads that will execute the
node ${v_i}$ on the same core serially (one after another) with no
other thread executing a different object on the core in between executions.

Given a timing-compositional architecture with the restrictions \textbf{R1} and \textbf{R2},
${c_i(n)}$ can be expressed for any node in terms of the memory demand of all instructions
of ${v_j \in V_i}$ into the cache ${\gamma_j}$ and the worst-case execution demand
to execute the node assuming all instructions of ${v_j}$ have
been cached ${\iota_j}$. Equation~\ref{eq:c_i} is an expression for
${c_i(n)}$. 
 
\begin{equation}
  \label{eq:c_i}
  c_i(n) = \begin{cases}
    0, & n \le 0 \\
    \gamma_j + \iota_j \cdot n, & n > 0
  \end{cases}
\end{equation}

For a node ${v_i \in V_j}$, the upper bound of memory demand of the node is denoted ${\gamma_i}$. It is the number of cycles required to load all blocks of the node. The complete set of blocks of the node are equivalent to the evicting cache blocks (ECBs)\addcite [Tan \& Mooney] of the object. Thus, the memory demand is the product of the BRT and count of ECBs of the node ${\textsc{ecb}_i}$ found in Equation~\ref{eq:mem-demand}.

\begin{equation}{\label{eq:mem-demand}}
    \gamma_i = \mathbb{B} \cdot \textsc{ecb}_i
\end{equation}

The execution demand ${\iota_i}$ for a node ${v_i \in V_i}$ is the worst case execution time of a single thread given all instructions are present in the cache. Any suitable WCET calculation method {\addcite} may be used to calculate the value.

%%
%% ct - commented out after rewriting in the paragraphs above, delete when you feel it
%%      is no longer pertinent.
%% 
%% \revise{For a high utilization task (i.e., task with utilization greater than 1), the
%% federated scheduler allocates a unique set of $m$ processors. Since there is no evictions
%% possible by other tasks, the term ${\gamma_i}$ is given by ${\mathbb{B}}$. In the
%% federated scheduler, all low utilization tasks share the remaining processors (i.e.,
%% processors not allocated to any high utilization task). Thus,   the The term ${\gamma_j}$
%% is the product of ${\mathbb{B}}$ and the number of ECBs (cite Tan and Mooney) in the node
%% ${\textsc{ecb}_j}$.}
%%

\section{Problem Formulation}
