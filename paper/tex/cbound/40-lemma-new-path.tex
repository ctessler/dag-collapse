Lemma~\ref{lemma:new-critical-path} provides a condition for beneficial
collapse when both nodes lie on the pre-collapse critical
path which creates a new critical path. Since the critical path length
is reduced by collapsing two nodes already lying upon it, it is
possible that a different path becomes the critical one after collapse.

\begin{lemma}[New Critical Path]\label{lemma:new-critical-path}
  When ${\Delta \ge 0}$, ${(u,v) \in \lambda_i}$, ${\hat{u} \not \in
    \hat{\lambda}_i}$, collapsing 
  ${u}$ and ${v}$ into ${\hat{u}}$ is beneficial if and only if:

  \begin{equation}
    \indent
    \frac{c_v(\eta_v)}{c_v(\eta_v) - \Delta} \ge
    \frac{C_i - L_i}{D_i - L_i}
  \end{equation}

  \begin{proof}
    After establishing a safe bound on the change in workload and
    critical path length the arithmetic manipulation used in
    Lemma~\ref{lemma:beneficial-both} is applied.

    Consider the pre-collapse DAG ${G_i}$, the critical path
    ${\lambda_i}$ has the greatest WCET sum among all paths through
    ${G_i}$. Name the pre-collapse path through ${G_i}$ with the
    second greatest WCET sum ${\lambda_i^\circ}$. Let ${L_i^\circ}$ denote the
    WCET sum of ${\lambda_i^\circ}$.

    For ${(u,v) \in \lambda_i}$ and ${\hat{u} \not \in
      \hat{\lambda}_i}$, then ${\lambda_i^\circ}$ must be the critical
    path post-collapse. Additionally, the reduction in the path length
    of ${\lambda_i}$ must be greater than the difference between
    ${L_i}$ and ${L_i^\circ}$:
    \begin{equation*}
      \indent
      L_i - \Delta < L_i^\circ \le L_i
    \end{equation*}

    Denote the difference in path lengths of ${\lambda_i}$ and
    ${\lambda_i^\circ}$ as ${\Delta^\circ = L_i - L_i^\circ}$. The
    post-collapse cores assigned to ${\tau_i}$ are:

    \begin{align*}
      \indent
      \frac{\hat{C_i} - \hat{L_i}} 
           {D_i - \hat{L_i}}
           &=
      \frac{C - \Delta - (L_i - \Delta^\circ)}
           {D_i - (L_i - \Delta^\circ)} \\
           &\le
      \frac{C - \Delta^\circ - (L_i - \Delta^\circ)}
           {D_i - (L_i - \Delta^\circ)}
    \end{align*}

    Reducing the workload by ${\Delta^\circ}$ in place of
    ${\Delta^\circ}$ increases the numerator because
    ${\Delta > \Delta^\circ}$. Reapplying the arithmetic manipulation
    of Lemma~\ref{lemma:beneficial-both}:

    \begin{align*}
      \frac{C - \Delta^\circ - (L_i - \Delta^\circ)}
           {D_i - (L_i - \Delta^\circ)}
           &\le
      \frac{C - L_i}
           {D_i - L_i}
    \end{align*}
  \end{proof}
\end{lemma}
