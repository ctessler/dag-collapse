\subsection{Relaxed Object Sizes}

In Theorem~\ref{thm:collapse}, all nodes ${v \in V}$ are restricted to
objects that fit entirely within the cache. This allows for the theorem
to use a simple closed form of the WCET for each node
${c_v(\eta_v)}$. The ultimate goal of the restriction is to provide the
simplest description within the theorem. This subsection will relax
the size restriction, allowing objects to exceed the size of the
cache. An alternative collapse condition will be presented in this
relaxed setting.

Under the relaxed setting, nodes retain their WCET as functions of the
number of threads. For a node ${v}$ the WCET of ${\eta_v}$ threads is
given by ${c_v(\eta_v)}$. This work assumes a suitable method for
calculating ${c_v(\eta_v)}$ such as \bundle{} or \bundlep{}
\addcite. Such methods may require more sophisticated scheduling
mechanisms for each node, rather than the strictly serialized approach 
used in the restricted setting.

Theorem~\ref{thm:relaxed-collapse}, is an abbreviation of
Theorm~\ref{thm:collapse}; stating the new condition and covering the
individual cases without revisiting the full structure of the
proof. It relies on the concept of the ``delta'' of execution bounds
determined by the WCET functions. 

\begin{definition}[WCET Bound Delta]
  For nodes ${u, v \in V}$ with the same executable object ${\alpha_u
    = \alpha_v}$, ${\Delta}$ is the difference in the separate
  execution times of ${\eta_u}$ and ${\eta_v}$ compared to the
  combined execution time. 

  ${\Delta = c_u(\eta_u) + c_v(\eta_v) - c_u(\eta_u + \eta_v)}$
\end{definition}


\begin{theorem}[Relaxed Conditional Collapse]\label{thm:relaxed-collapse}
  For a task ${\tau_i \in \tau}$, associated DAG ${G_i \in \mathbb{G}}$,
  where ${G_i = (V, E)}$, candidate nodes ${u,v \in V}$,
  collapsing ${u}$ and ${v}$ into ${\hat{u}}$ is beneficial if:

  \begin{equation}
    \Delta > 0
  \end{equation}

  and
  
  \begin{equation}
    \indent
    \frac{c_v(\eta_v)}{c_v(\eta_v) - \Delta} \ge
    \frac{C_i - L_i}{D_i - L_i}
  \end{equation}

  \begin{proof}
    The four cases that satisfy Equation~\ref{eq:collapse-condition}
    are enumerated below.

    \setcounter{case}{0}
    \begin{case}
      [${(u, v \not \in \lambda_i) \land (\hat{u} \not \in \hat{\lambda_i})}$]
      Only ${C_i}$ is affected when no node ${(u,v,\hat{u})}$ lie on
      the critical path. Thus:
      \begin{align}
        \indent
        \hat{C}_i = C_i - \Delta
      \end{align}

      Since ${\Delta}$ must be greater than zero, then ${\hat{C}_i}$ is
      less than ${C_i}$ and \ref{eq:collapse-condition} is satisfied.
    \end{case}

    \begin{case}
      [${(u \in \lambda_i \land v \not \in \lambda_i)
          \land (\hat{u} \in \hat{\lambda_i})}$]
      Collapsing ${v}$ into the critical path which ${\hat{u}}$ lies
      on, increases the critical path length ${L_i}$. Quantified, the
      increase in ${L_i}$ is the removal of ${c_u(\eta_u)}$ and the
      addition of ${c_u(\eta_u + \eta_v)}$ which is equivalent to:
      \begin{align}
        \indent
        \hat{L}_i = L_i + c_v(\eta_v) - \Delta
      \end{align}

      Assuming Equation~\ref{eq:collapse-condition} is true, the
      condition is derived by:
      \begin{align*}
        \frac{\hat{C_i} - \hat{L_i}} 
             {D_i - \hat{L_i}} &\le
        \frac{C_i - L_i}
             {D_i - L_i} \\
        \frac{(C_i - \Delta) - (L_i + c_v(\eta_v) - \Delta)}
             {D_i - (L_i + c_v(\eta_v) - \Delta)} & \le
        \frac{C_i - L_i}
             {D_i - L_i} \\
        (D_i - L_i)(C_i - L_i) - (D_i - L_i)c_v(\eta_v) & \le
             (C_i - L_i)(D_i - L_i) - (C_i - L_i)(c_v(\eta_v) - \Delta) \\
        \frac{c_v(\eta_v)}{c_v(\eta_v) - \Delta} & \ge
        \frac{C_i - L_i}{D_i - L_i}
      \end{align*}
    \end{case}

    \begin{case}[${(u,v \in \lambda_i) \land (\hat{u} \in \hat{\lambda_i})}$]
      When both nodes lie on the critical path, the critical path
      length after collapse is:
      \begin{align*}
        \indent
        \hat{L}_i = L_i - \Delta
      \end{align*}

      Substituting the new critical path length and workload into
      Equation~\ref{eq:collapse-condition}:

      \begin{align*}
        \indent
        \frac{\hat{C_i} - \hat{L_i}} 
             {D_i - \hat{L_i}} &\le
        \frac{C_i - L_i} 
             {D_i - L_i} \\
        \frac{C_i - \Delta - (L_i - \Delta)}
             {D_i - (L_i - \Delta)}
             &\le
        \frac{C_i - L_i} 
             {D_i - L_i} \\
        \frac{C_i - L_i}
             {D_i - L_i + \Delta}
             &\le
        \frac{C_i - L_i} 
             {D_i - L_i}
      \end{align*}

      Since ${\Delta}$ is greater than zero, the inequality holds.
    \end{case}

    \begin{case}[${(u,v \not \in \lambda_i) \land (\hat{u} \in \hat{\lambda_i})}$]
      When a new critical path is created, the length of the new
      critical path is bounded by

      \begin{align*}
        \indent
        \hat{L}_i - L_i &\le \Delta - c_u(\eta_u) \\
        \hat{L}_i - L_i &\le c_v(\eta_v) - \Delta \\
        \hat{L}_i &\le L_i + c_v(\eta_v) - \Delta \\
      \end{align*}

      Substituting the inequality of
      ${\hat{L}_i \le L_i + c_v(\eta_v) - \Delta}$ derived here in
      Case 4, into Case 2 reaches the same conclusion. If
      Equation~\ref{eq:condition} is satisfied then
      Equation~\ref{eq:collapse-condition} is true.
    \end{case}

    Given ${\Delta}$ is positive, and
    Equation~\ref{eq:collapse-condition} is true, each of the four
    cases demonstrate a benefit for collapsing ${u}$ and ${v}$.
  \end{proof}
\end{theorem}
