Lemma~\ref{lemma:beneficial-delta} addresses the simplest case of
collapse, where ${u}$ and ${v}$ do not lie on the pre-collapse
critical path and ${\hat{u}}$ does not lie on the post-collapse
critical path. Since the path is not affected the critical path
length remains the same.

\begin{lemma}[Beneficial Delta]\label{lemma:beneficial-delta}
  When ${(u, v) \not \in \lambda_i}$, ${\hat{u} \not \in \hat{\lambda}_i}$
  collapsing ${u}$ and ${v}$ into ${\hat{u}}$ is beneficial if and
  only if ${\Delta \ge 0}$.

  \begin{proof}
    Since the critical path is not affected, neither is the critical
    path length: ${\hat{L}_i = L_i}$. The workload post-collapse
    ${\hat{C}_i}$ is the pre-collapse workload minus the workload
    delta of collapse:
    \begin{equation*}
      \indent
      \hat{C}_i = C_i - \Delta
    \end{equation*}
    Replacing ${\hat{C}_i}$ and ${\hat{L}_i}$ in Equation~\ref{eq:benefit}:
    \begin{equation*}
    \frac{(\hat{C_i} - \Delta) - L_i}
           {D_i - L_i} \le
      \frac{C_i - L_i}
           {D_i - L_i}
    \end{equation*}
    If ${\Delta}$ is positive, the inequality will hold and collapse
    is beneficial. Similarly, if collapse is beneficial and
    ${\hat{L}_i = L}$, ${\Delta}$ must be positive.
  \end{proof}
\end{lemma}

