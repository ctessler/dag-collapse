\section{Collapsing of Nodes}
\label{sec:collapse-bound}

For a task ${\task{i} \in \tasks{}}$, associated DAG
${\Dag{i} \in \dagset{}}$, where ${\Dag{i} = (V, E)}$, with nodes
${u,v \in V}$ that qualify as candidates for collapse, collapsing
${u}$ and ${v}$ may (or may not) result an increase in
performance. This section is provides a method of determining if the
collapse of ${u}$ and ${v}$ will \emph{benefit} performance of the
task set ${\tasks{}}$ in terms of ${m_i}$: the number of cores
dedicated to \task{i}.

For a task \task{i} with \noofcores{i} dedicated cores, collapsing
${u}$ and ${v}$ benefits \task{i} (and therefor \tasks{}) if the
number of cores dedicated to \task{i} is potentially reduced. If
\noofcores{i} is the number of cores dedicated to \task{i} before
collapsing ${u}$ and ${v}$, and ${\hat{\noofcores{i}}}$ the number of
cores after collapse, beneficial collapse is defined as follows: 

\begin{definition}[Beneficial Collapse]
  Collapsing ${u}$ and ${v}$ is \emph{beneficial} if and only if
  ${\hat{\noofcores{i}} \le \noofcores{i}}$.
\end{definition}

The number of cores dedicated to task \task{i} is determined by
Equation~\ref{eq:m} which is replicated below. Equation~\ref{eq:m}
depends on the critical path length \criticalpathlen{i} and workload
\workload{i}, both of which can be affected by collapsing nodes.

\begin{multicols}{2}
  \begin{equation*}
    m_i = \left\lceil
      \frac{\workload{i} - \criticalpathlen{i}}
           {\Deadline{i} - \criticalpathlen{i}}
    \right\rceil
  \end{equation*}
  \begin{equation*}
    \workload{i} = \sum_{v \in V} c(v)
  \end{equation*}
\end{multicols}

For simplicity of presentation, it is assumed the entire object each
thread assigned to ${v}$ executes fits entirely within the
cache of each core. Additionally, each core is required to be
timing-compositional~\addcite{}, where the memory and execution demand
are separable. The memory demand of populating the cache with the
entire object associated with ${v}$ is referred to as
${\mathbb{B}_v}$. Execution of one thread over the object of ${v}$ is
referred to as ${\mathbb{I}_v}$. The number of threads executed over
the object of ${v}$ scheduled one core of is referred to as
${|v|}$.

Applying the principle of inter-thread cache benefit from \addcite{}
[BUNDLE and BUNDLEP], the WCET of a node ${v}$ is given by:

\begin{equation}
  c(v) = \mathbb{B}_v + |v| \cdot \mathbb{I}_v
\end{equation}

When comparing parameters before and after collapse, the after
collapse values will be give a ``hat'' indicator. For example, the
critical path length before collapse is denoted
\criticalpathlen{i}; after collapse, it is denoted
${\hat{\criticalpathlen{i}}}$.

When two nodes ${u,v \in V}$ are collapsed into ${\hat{u}}$, the WCET
of ${\hat{u}}$ can be described in terms of the WCET of ${u}$ and
${v}$. Under the assumption that ${u}$ and ${v}$'s object fits
entirely in the cache, combined with the timing-compositional
requirement of each core, scheduling ${|u|}$ followed by ${|v|}$
threads on the same core has a memory demand of
${\mathbb{B}_u}$. The execution demand of |u| followed by |v| threads
is ${(|u| + |v|) \cdot \mathbb{I}_u}$ because ${\mathbb{I}_u =
  \mathbb{I}_v}$. Thus, the WCET of ${\hat{c}(u)}$ can be defined as 
follows:
\begin{equation*}
  \hat{c}(u) = c(u + v) = \mathbb{B}_u + |u + v| \cdot \mathbb{I}_u
  \label{eq:restriction}
\end{equation*}

\begin{theorem}[Conditional Collapse] For a task
  ${\task{i} \in \tasks{}}$, associated DAG ${\Dag{i} \in \dagset}$,
  where ${\Dag{i} = (V, E)}$, candidate nodes ${u,v \in V}$,
  collapsing ${u}$ and ${v}$ into ${u}$ is beneficial if:

  \begin{equation}
    \indent
    1 + \frac{\mathbb{B}_u}{\mathbb{I}_u}
    >
    \frac{\workload{i} - \criticalpathlen{i}}
         {\Deadline{i} - \criticalpathlen{i}}
    \label{eq:condition}
  \end{equation}

  \begin{proof}
    The goal of the proof is to show the comparison of
    Equation~\ref{eq:collapse-condition} is true
    under all circumstances. There are four cases which cover all
    possible changes to the critical path length and workload of the
    task \task{i}. Note, the deadline of the task \Deadline{i} cannot
    be changed and does not need consideration. 

    \begin{equation} \label{eq:collapse-condition}
      \indent
      \frac{\hat{\workload{i}} - \hat{\criticalpathlen{i}}}
           {\Deadline{i} - \hat{\criticalpathlen{i}}} \le
      \frac{\workload{i} - \criticalpathlen{i}}
           {\Deadline{i} - \criticalpathlen{i}}
    \end{equation}

    In summary, the four cases to consider are:

    \begin{enumerate}
    \item ${(u, v \not \in \criticalpath{i})
      \land
      (\hat{u}, \hat{v} \not \in \hat{\criticalpath{i}})}$:
      Neither node is on the critical path
      before or after collapse.
    \item ${(u \in \criticalpath{i} \land v \not \in \criticalpath{i})
      \land
      (\hat{u} \in \hat{\criticalpath{i}})}$: One node (call it ${u}$)
      is on the critical path before collapse and remains on the path
      after collapse.
    \item ${(u,v \in \criticalpath{i}) \land
      (\hat{u},\hat{v} \in \hat{\criticalpath{i}})}$: Both nodes lie
      on the critical path before and after collapse.
    \item ${(u,v \not \in \criticalpath{i}) \land
      (\hat{u},\hat{v} \in \hat{\criticalpath{i}})}$: Neither node
      belongs to the critical path before collapse. Collapsing the
      nodes creates a \textbf{new} critical path.
    \end{enumerate}

    \begin{case}[${(u, v \not \in \criticalpath{i}) \land
          (\hat{u}, \hat{v} \not \in \hat{\criticalpath{i}})}$] By
      definition of critical path and critical path length, ${u}$ and
      ${v}$ do not contribute to \criticalpathlen{i} nor to
      ${\hat{\criticalpathlen{i}}}$. Therefor,
      ${\criticalpathlen{i} = \hat{\criticalpathlen{i}}}$. The
      workload has been affected by the removal of ${v}$ and ${u}$'s execution
      and the increase in ${\hat{u}}$'s execution. 
      \begin{align}
        \indent
        \hat{C}_i &= C_i - c(u) - c(v) + \hat{c}(u) \\
        &= C_i - \mathbb{B}_u - |u| \cdot \mathbb{I}_u
            - \mathbb{B}_v - |v| \cdot \mathbb{I}_v + \hat{c}(u)
            \label{eq:pre-u} \\
        &= C_i - \mathbb{B}_u - |u| \cdot \mathbb{I}_u
            - \mathbb{B}_u - |v| \cdot \mathbb{I}_u + \hat{c}(u)
            \label{eq:post-u} \\
        &= C_i -2\mathbb{B}_u - |u + v|\cdot \mathbb{I}_u +
            \hat{c}(u) \\
        &= C_i -2\mathbb{B}_u - |u + v|\cdot \mathbb{I}_u +
            \mathbb{B}_u + |u + v|\cdot \mathbb{I}_u \\
        &= C_i - \mathbb{B}_u
      \end{align}

      From Equation~\ref{eq:pre-u} to Equation~\ref{eq:post-u}, the
      subscripts on the memory and execution demand of ${v}$ are
      transferred to ${u}$. This is permissible because ${u}$ and
      ${v}$ refer to the same object which ${|u + v|}$ threads
      execute.

      It must be the case that ${\mathbb{B}_u \ge 0}$, therefore
      ${\hat{C}_i \le C_i}$, the condition of
      Equation~\ref{eq:condition} is true.
    \end{case}

    \begin{case}[${(u \in \criticalpath{i} \land v \not \in \criticalpath{i})
      \land
      (\hat{u} \in \hat{\criticalpath{i}})}$]
    \end{case}

    Cases 1-4 encompass all conditions under which candidates for
    collapse may impact the number of cores assigned to \task{i}. In
    each circumstance, condition~\ref{eq:condition} is true. Thus, for
    candidate nodes obeying the restriction~\ref{eq:restriction} it is
    beneficial to collapse them. 
  \end{proof}
\end{theorem}
