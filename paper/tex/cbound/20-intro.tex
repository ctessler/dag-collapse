\section{Electing Nodes for Collapse}
\label{sec:collapse-bound}

For a task ${\tau_i \in \tau}$, associated DAG
${G_i \in \mathbb{G}}$, where ${G_i = (V, E)}$, with nodes
${u,v \in V}$ that qualify as candidates for collapse, collapsing
${u}$ and ${v}$ may (or may not) result in an increase in
performance. This section provides a method of determining if the
collapse of ${u}$ and ${v}$ will \emph{benefit} the performance of the 
task set ${\tau}$ in terms of ${m_i}$: the number of cores
dedicated to ${\tau_i}$.

The number of cores dedicated to task ${\tau_i}$ is determined by
Equation~\ref{eq:m} which is replicated below. Equation~\ref{eq:m}
depends on the critical path length ${L_i}$ and workload
${C_i}$, both of which can be affected by collapsing nodes.

\begin{multicols}{2}
  \begin{equation*}
    m_i = \left\lceil
      \frac{C_i - L_i}
           {D_i - L_i}
    \right\rceil
  \end{equation*}
  \begin{equation*}
    C_i = \sum_{v \in V} c_v(\eta_v)
  \end{equation*}
\end{multicols}

For a task ${\tau_i}$ with ${m_i}$ dedicated cores, collapsing
${u}$ and ${v}$ benefits ${\tau_i}$ (and therefor ${\tau}$) if the
number of cores dedicated to ${\tau_i}$ is potentially reduced. A
``hat'' is used to indicate pre and post-collapse values,
e.g. ${\hat{C}_i}$ is the workload post-collapse and ${C_i}$
pre-collapse. If ${m_i}$ is the number of cores dedicated to
${\tau_i}$ by Equation~\ref{eq:m} before collapsing ${u}$ and ${v}$
beneficial collapse is defined as follows:  

\begin{definition}[Beneficial Collapse]
  Collapsing ${u}$ and ${v}$ is \emph{beneficial} if and only if
  ${\hat{m_i} \le m_i}$. 

  Which can be expanded to:
  \begin{equation}\label{eq:benefit}
      \frac{\hat{C_i} - \hat{L_i}}
           {D_i - \hat{L_i}} \le
      \frac{C_i - L_i}
           {D_i - L_i}
  \end{equation}
\end{definition}


Section~\ref{sec:candidacy} defines the candidates for
collapse based upon the structure of the DAG. The goal of this section
is to provide a condition that determines if candidate nodes ${u}$ and
${v}$ will benefit the task ${\tau_i}$. If ${u}$ and ${v}$ satisfy the
condition collapsing them  is beneficial and the nodes are
\emph{elected} for collapse.

The final Theorem~\ref{thm:election-condition}, relies on smaller
lemmas to provide a single condition for election. Each of the lemmas,
definitions, and theorems share the following context. A single task
${\tau_i \in \tau}$ with associated DAG ${G_i \in \mathbb{G}}$, where
${G_i = (V, E)}$, and nodes ${u,v \in V}$ are candidates for collapse
according to Section~\ref{sec:candidacy}. Additionally, the collapsed
node of ${u}$ and ${v}$ will be referred to as ${\hat{u}}$ which may
be stated as ``collapsing ${u}$ and ${v}$ \emph{into}
${\hat{u}}$''. For simplicity of discussion, if one of ${u}$ and ${v}$
must lie on the critical path and the other must not, the node on the
critical path is named ${u}$.

The WCET of the collapsed node ${\hat{u}}$ is given by
${c_{\hat{u}}(\eta_{\hat{u}})}$ which is equal to
${c_{\hat{u}}(\eta_{\hat{u}}) = c_u(\eta_u + \eta_v)}$. Quantification
of the difference in WCET of the separate and combined WCET times is
called the workload delta and represented as ${\Delta}$.

\begin{definition}[Workload Delta]
  ${\Delta}$ is the difference in the separate WCET bounds of
  ${u}$ and ${v}$ compared to the combined execution time. 

  ${\Delta = c_u(\eta_u) + c_v(\eta_v) - c_u(\eta_u + \eta_v)}$
\end{definition}

Each of the following lemmas are related to collapsing ${u}$ and ${v}$
for each of the potential changes to the critical path. Individual
lemmas will provide conditions for the collapse to be beneficial given
the impact on the critical path. Candidacy limits the impact on the
critical path length to one of the following:

\begin{enumerate}
  \item{${u}$ and ${v}$ did not lie on the pre-collapse critical path and
    ${\hat{u}}$ does not lie on the post-collapse critical path:
    ${(u,v) \not \in \lambda_i}$ and ${\hat{u} \not \in \hat{\lambda}_i}$.}
  \item{${u}$ lies on the pre-collapse critical path, ${v}$ does
    not, and ${\hat{u}}$ lies on the post-collapse critical path:
    ${u \in \lambda \land v \not \in \lambda_i}$
    and ${\hat{u} \in \hat{\lambda}_i}$.}
    
  %% \item{${u}$ lies on the pre-collapse critical path, ${v}$ does not,
  %%   and ${\hat{u}}$ does not lie on the post-collapse critical path:
  %%   ${u \in \lambda_i \land v \not \in \lambda_i}$
  %%   and ${\hat{u} \not \in \hat{\lambda}_i}$.}
    
  \item{${u}$ and ${v}$ lie on the pre-collapse critical path,
    and ${\hat{u}}$ lies on the post-collapse critical path:
    ${(u, v) \in \lambda_i}$ and ${\hat{u} \in \hat{\lambda}_i}$.}
  \item{${u}$ and ${v}$ lie on the pre-collapse critical path, and
    ${\hat{u}}$ does not lie on the post-collapse critical path:
    ${(u,v) \in \lambda_i}$ and ${\hat{u} \not \in \hat{\lambda}_i}$.}
\end{enumerate}

\input{\tex{25-lemma-delta.tex}}
\input{\tex{30-lemma-half.tex}}
\input{\tex{35-lemma-both.tex}}
\input{\tex{40-lemma-new-path.tex}}
\input{\tex{45-theorem-collapse.tex}}
