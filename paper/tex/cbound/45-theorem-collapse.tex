Combining the four lemmas provides a set of conditions in
Theorem~\ref{thm:election-condition} that elect nodes for collapse.

\begin{theorem}[Election]\label{thm:election-condition}
  Collapsing candidate nodes ${(u,v)}$ is beneficial if and only if

  \begin{multicols}{2}
    \begin{equation*}
      \Delta \ge 0
    \end{equation*}
    \begin{equation*}
      \frac{c_v(\eta_v)}{c_v(\eta_v) - \Delta} \ge
      \frac{C_i - L_i}{D_i - L_i}
    \end{equation*}
  \end{multicols}

  \begin{proof}
    A direct application of Lemmas~\ref{lemma:beneficial-delta},
    \ref{lemma:beneficial-onto}, \ref{lemma:beneficial-both},
    and \ref{lemma:new-critical-path}. Candidacy restricts the
    impact upon the workload and critical path length of collapse to
    the cases described by the four lemmas. Since ${u}$ and ${v}$ are
    arbitrary candidates all conditions required by the lemmas are
    required to guarantee collapse is beneficial. Those conditions are
    ${\Delta \ge 0}$ and ${\frac{c_v(\eta_v)}{c_v(\eta_v) - \Delta} \ge
      \frac{C_i - L_i}{D_i - L_i}}$. If those conditions are met, then
    collapse is beneficial.
  \end{proof}
\end{theorem}
