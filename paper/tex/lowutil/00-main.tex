\section{Low Utilization Tasks}

In the federating scheduling setting of this work, collapsing nodes of
high utilization tasks may reduce the number of cores dedicated to
those individual tasks. The reduction is achieved by accounting for
the inter-thread cache benefit between threads of the task and
scheduling in a cache cognizant manner. Bringing the inter-thread
cache benefit to low utilization tasks is outside the scope of this
work, and is left for consideration in later work.

Low utilization tasks are serialized, their DAGs are converted to a
graph where nodes have in and out-degree of at most one. Each low
utilization task is converted to a single block of execution,
scheduled alongside the other low utilization tasks. Preemptions
between low utilization tasks are permitted by the selected scheduling
algorithm algorithm such as earliest deadline first (EDF). The
collapsing process described herein does not apply when preemptions
are permitted during the execution of a node. Nor is the proposed
model compatible with known cache-aware limited preemption techniques
such as \addcite{} [Bertogna].

To account for the impact of caches upon low utilization tasks, two
analyses are required. First, the WCET of each serialized task must
include a cache analysis \addcite{} [Arnold, Mueller, Something more
  modern]. Second, due to the possibility of preemptions cache related
preemption delay analysis \addcite{} [Altmeyer, etc.] must be
performed for the task set under the given scheduling algorithm
\addcite{} [Luniss].
